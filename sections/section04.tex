\section{Version control of artistic assets}
At this point, the game is already working with the most important features implemented. Following the 
\href{https://docs.godotengine.org/en/stable/getting_started/first_2d_game/06.heads_up_display.html}{\color{blue}guide},
there are two additional features contributing to the experience, namely the heads-up display and sound effects, however,
the purpose of this paper is to demonstrate version control application \textbf{Helix Core}\textsuperscript{\texttrademark}
which is already fulfilled. There remains only one outstanding point, the version control of animations and other artistic
assets used for the game. This scenario plays important role when the game development includes also production of own
assets and artistic teams are also working on animations and 3D models. \hfill \break
For demonstration purposes, the sprites used for the game will be amended and see what version control techniques are 
available in \textbf{Helix Core}\textsuperscript{\texttrademark} to keep track of change.
\begin{itemize}
    \item create new stream for artistic team {$=>$} call this \textit{art1.0}
    \item observe how the development branches (streams) interact with the main branch, e.g. changes in artistic elements
    copied to main and from there, merged to the other development stream used by game programmers.
\end{itemize}