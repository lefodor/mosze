\section{Version control using Helix Core\textsuperscript{\texttrademark}} \label{case-study-1}
In this section, a very basic 2D game is created with the game engine Godot. The development process is version 
controlled using the Helix Core application to demonstrate the major steps and show the differences compared to 
traditional SW development workflow.
\begin{itemize}
    \item centralized version control system: files are maintained on centralized server
    \item files are stored in depots: typically 1 depot per project
    \item workspace: mapping of relevant files between local machine and server
\end{itemize}

\subsection{Setup Helix Core}
The setup of Helix Core follows the instructions from the 
\href{https://help.perforce.com/helix-core/quickstart/Content/quickstart/admin-install-linux.html}{\color{blue}{Helix Core Server Administrator Guide}}
(or in \cite{helix} ) relevant for Linux (Ubuntu) systems. The following commands were issued for setting up the version control application, as 
can be found in the aformentioned documentation.
\begin{enumerate}
    \item Download perforce public key
    \begin{verbatim}
        $ wget https://package.perforce.com/perforce.pubkey
    \end{verbatim}
    \item Obtain the fingerprint of the public key and verify
    \begin{verbatim}
        $ gpg -n --import --import-options import-show perforce.pubkey
        $ gpg -n --import --import-options import-show perforce.pubkey | 
        grep -q "E58131C0AEA7B082C6DC4C937123CB760FF18869" && echo "true"
    \end{verbatim}
    \item Add the public key to your keyring
    \begin{verbatim}
        $ wget -qO - https://package.perforce.com/perforce.pubkey | 
        sudo apt-key add -
    \end{verbatim}
    \item Create a new file for the Perforce repository
    \begin{verbatim}
        $ sudo nano /etc/apt/sources.list.d/perforce.list
    \end{verbatim}
    \item In the new file, input the following line
    \begin{verbatim}
        deb http://package.perforce.com/apt/ubuntu focal release
    \end{verbatim}
    Make sure the version matches the Linux/Ubuntu version of the machine's system. E.g 'focal' needs to be replaced by
    'jammy' if the most recent Linux/Ubuntu LTS version is running on the machine.
    \item Update machine and install Helix Core
    \begin{verbatim}
        $ sudo apt-get update
        $ sudo apt-get install helix-p4d
    \end{verbatim}
    \item Run configure file
    \begin{verbatim}
        $ sudo /opt/perforce/sbin/configure-helix-p4d.sh
    \end{verbatim}
    During the installation process, I had problems with this step when I tried to remove and install the helix-p4d
    package. The correct way of removing all related data is summarised as follows:
    \begin{itemize}
        \item Remove package
        \begin{verbatim}
            $ sudo apt-get remove helix-p4dctl
            $ sudo apt-get purge helix-p4dctl
        \end{verbatim}
        \item For some reasons, the file \colorbox{blue!30}{/etc/perforce/p4dctl.conf.d/p4d.template} was missing after
        these steps. The file was recreated with below content (downloaded from \href{https://portal.perforce.com/s/article/15056}{\color{blue}{here}} 
        and slightly modified) and the install process worked again.
        \begin{verbatim}
            p4d %NAME%
            {
                Owner    =    perforce
                Execute  =    /opt/perforce/sbin/p4d
                Umask    =    077

                # Enabled by default.
                Enabled  =    true

                Environment
                {
                    P4ROOT    =    %ROOT%
                    P4SSLDIR  =    ssl
                    PATH      =    /bin:/usr/bin:/usr/local/bin:...
                    /opt/perforce/bin:/opt/perforce/sbin

                # Enables nightly checkpoint routine
                # This should *not* be considered a complete backup...
                solution
                MAINTENANCE =     true
                }
            }
        \end{verbatim}
    \end{itemize}
    \item Download visual interface (p4v) from 
    \href{https://www.perforce.com/downloads/helix-visual-client-p4v}{\color{blue} Download Page}.
    P4v is the client application commnicating with the server. Server in this case is created on local machine. In case of
    reinstall, the P4V client might also need to be unpacked again so that the already created workspaces do not block
    the execution. For more information, check \cite{p4v}.
    \item Additional config steps
    \begin{verbatim}
        $ export P4PORT=ssl:1666
        $ export P4USER=super
        # add the following line to ~/.bashrc
        export PATH=${PATH}:~/perforce/p4v-2023.3.2495381/bin
    \end{verbatim}
    \item Login and start p4v
    \begin{verbatim}
        $ p4 login
        $ p4v # launches GUI
    \end{verbatim}
\end{enumerate}
Having done the above steps, the GUI can be launched bringing up the screen below. The location of the server (where
all shared files/projects are stored) \colorbox{blue!30}{/opt/perforce/servers/master/root} 
As the server is also created on local machine. This folder is used to create the mapping between server and other 
client machines and workspaces.
\begin{figure}[H]
    \centering
    \includegraphics[width=\textwidth]{p4vlogin20231029.png}
      \caption{p4v login}
      \label{fig:p4v login}
\end{figure}

\subsection{Setup Godot}
Setting up Godot game engine is relatively easy compared to other game engines like Unity\textsuperscript{\texttrademark}
or Unreal Engine\textsuperscript{\texttrademark}. Godot is lightweight and there are basically two options to get hold
of the application:
\begin{itemize}
    \item Download pre-build application specific to host machine OS and programming language (GDScript or .Net)
    \href{https://godotengine.org/download/linux/}{\color{blue}Download page}
    \item Build from source following instructions 
    \href{https://docs.godotengine.org/en/stable/contributing/development/compiling/index.html}{\color{blue}Build from source}.
\end{itemize}
For the purpose of this paper the first approach was adopted by downloading the Godot 4.1.2 stable version for Linux.
After downloading and unpacking, Godot is ready to use:
\begin{figure}[H]
    \centering
    \includegraphics[width=\textwidth]{godotopen.png}
      \caption{godot editor}
      \label{fig:godot}
\end{figure}
The Godot Engine documentation including tutorials can be found in \cite{godot}.