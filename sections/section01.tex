\section{Overview of version control concepts}
The main goals of version control systems can be summarized as follows:
\begin{itemize}
    \item keep track of change history. By keeping the history of changes on the code base, it becomes possible to revert
    to a previous state. Furthermore, the identification of bugs' first occurrence and the changeset that introduced them
    will also be possible.
    \item maintain several version of the code, e.g. PRODUCTION, RELEASE CANDIDATE, DEVELOPMENT. This is particulary
    useful when some features have already been developed but for some reasons they are planned to be released later in 
    the future.
    \item allow several people/teams working on the same code base simultaneously. Currently, any commercial software
    product requires the effective collaboration of many programmers (even hundreds), it would become swiftly chaotic
    even impossible without a tool that can handle code changes from so many sources.
\end{itemize}
To satisfy the above mentioned needs, version control applications have been developed that differ from each other in
some or many aspects. \\
The most straightforward distinction between version control systems is based on whether they treat code repository in 
a centralized or distributed manner. The former designates an authorative data store and all activies/changes are 
aligned with reference to this central hub. In the latter case, however, there is no authorative central repository.
Code repositories are aligned by activies called merges when changes are combined creating a new repository version 
incorporating all modifications of the code base. \\
Centralized version control system (CVCS) usually implements the following workflow:
\begin{itemize}
    \item check out code repository from central location (usually from server) with all the changes made by other
    colleagues
    \item implement own changes
    \item commit changes, in other words, upload own changes so that others can work with the most up to date version of
    the repository.
\end{itemize}
DVCS works with the following typical workflow:
\begin{itemize}
    \item clone repository with full history of changes
    \item make own changes
    \item commit changes to local repository
    \item push changes to DVCS
\end{itemize}
    \subsection{Subsection01}
    ...

    \subsection{Subsection02}
    ...

    \subsubsection{Subsubsection $\Delta$}
    ...
