\section{Overview of version control concepts} \label{overview-version-control}
The main goals of version control systems can be summarized as follows:
\begin{itemize}
    \item keep track of change history. By keeping the history of changes on the code base, it becomes possible to revert
    to a previous state. Furthermore, the identification of bugs' first occurrence and the changeset that introduced them
    will also be possible.
    \item maintain several version of the code, e.g. \textbf{production}, \textbf{release candidate}, \textbf{development}. 
    This is particulary useful when some features have already been developed but for some reasons they are planned to be 
    released later in the future.
    \item allow several people/teams working on the same code base simultaneously. Currently, any commercial software
    product requires the effective collaboration of many programmers (even hundreds), it would become swiftly chaotic
    even impossible without a tool that can handle code changes from so many sources.
\end{itemize}
To satisfy the above mentioned needs, version control applications have been developed that differ from each other in
some or many aspects. \\
The most straightforward distinction between version control systems is based on whether they treat code repository in 
a centralized or distributed manner. The former designates an authorative data store and all activies/changes are 
aligned with reference to this central hub. In the latter case, however, there is no authorative central repository.
Code repositories are aligned by activities called merges when changes are combined creating a new repository version 
incorporating all modifications of the code base. \\
Centralized version control system (CVCS) usually implements the following workflow \cite{atlassian-vcs}:
\begin{itemize}
    \item check out code repository from central location (usually from server) with all the changes made by other
    colleagues
    \item implement own changes
    \item commit changes, in other words, upload own changes so that others can work with the most up to date version of
    the repository.
\end{itemize}
DVCS works with the following typical workflow:
\begin{itemize}
    \item clone repository with full history of changes
    \item make own changes
    \item commit changes to local repository
    \item push changes to DVCS and merge to main branch related to the activities
\end{itemize}
Main risk of centralized VCS is related to the authorative repository. If the hosting environment of this repository is
unavailable for whatever reasons, it can jeopardize productivity or even cause data loss. The distributed architecture
is less prone to such events as it keeps the whole code base at each machine that checks out the repository.
\subsection{Branching strategy}
Another important aspect of version control is the applied method to keep and maintain different versions of the code base.
As mentioned previously, different code versions/branches can be summarized as follows:
\begin{itemize}
    \item \textbf{main} (production): released version of the code, currently in production or in live environment
    \item \textbf{release branch/release candidate}: in a periodic development cycle, there is usually an event, the so-called 
    code freeze, after which a release branch is created that already contains all incremental changes of the software 
    that are planned to go-live with upcoming release. All other changes, that are not part of next release but already
    in progress stay in the development branch. After the release, the release branch can be archived. There can also
    be situations when issues are identified after go-live and a hotfix needs to be issued using a \textbf{hotfix} branch.
    \item \textbf{development} branch: all changes done by developers start in a development phase (development branch) 
    and mature until they are tested and become ready to go-live (or get cancelled and removed from the repository).
    Depending on the branching strategy and development situation, there can be one or several development branches,
    including \textbf{feature} branches that are specific to a new functionality and represents a more extensive workload before
    it can be released.
\end{itemize}
An important part of development workflow is the method used to create and organize these branches, execute merges. These
activites are governed by the so-called branching strategy. There are several branching strategies that can be adopted,
the final decision always depends on the nature of development project, available tools (e.g. ticketing system) etc.
For further details can be found about Git branching strategy and workflow at \cite{gitflow-workflow}, or Microsoft's
practice at \cite{ms-tfs-branching}, whereas Perforce's tool - which will be demonstrated in later sections - at 
\cite{perforce-branching}.