\section{Overview of version control concepts}
The most straightforward distinction between version control systems is based on whether they treat code repository in 
a centralized or distributed manner. The former designates an authorative data store and all activies/changes are 
aligned with reference to this central hub. In the latter case, however, there is no authorative central repository.
Code repositories are aligned by activies called merges.  
Centralized version control system (CVCS) usually implements with the following workflow:
\begin{itemize}
    \item check out code repository from central location (usually from server) with all the changes made by other
    colleagues
    \item implement own changes
    \item commit changes, in other words, upload own changes so that others can work with the most up to date version of
    the repository.
\end{itemize}
    \subsection{Subsection01}
    ...

    \subsection{Subsection02}
    ...

    \subsubsection{Subsubsection $\Delta$}
    ...
