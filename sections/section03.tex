\section{Version control using Helix Core\textsuperscript{\texttrademark}}
In this section, a very basic 2D game is created with the game engine Godot. The development process is version 
controlled using the Helix Core application to demonstrate the major steps and show the differences compared to 
traditional SW development workflow.

\subsection{Setup Helix Core and Godot}
The setup of Helix Core follows the instructions from the 
\href{https://help.perforce.com/helix-core/quickstart/Content/quickstart/admin-install-linux.html}{\color{blue}{Helix Core Server Administrator Guide}}
relevant for Linux (Ubuntu) systems. The following commands were issued for setting up the version control application, as 
can be found in the aformentioned documentation.
\begin{enumerate}
    \item Download perforce public key
    \begin{verbatim}
        $ wget https://package.perforce.com/perforce.pubkey
    \end{verbatim}
    \item Obtain the fingerprint of the public key and verify
    \begin{verbatim}
        $ gpg -n --import --import-options import-show perforce.pubkey
        $ gpg -n --import --import-options import-show perforce.pubkey | 
        grep -q "E58131C0AEA7B082C6DC4C937123CB760FF18869" && echo "true"
    \end{verbatim}
    \item Add the public key to your keyring
    \begin{verbatim}
        $ wget -qO - https://package.perforce.com/perforce.pubkey | 
        sudo apt-key add -
    \end{verbatim}
    \item Create a new file for the Perforce repository
    \begin{verbatim}
        $ sudo nano /etc/apt/sources.list.d/perforce.list
    \end{verbatim}
    \item In the new file, input the following line
    \begin{verbatim}
        deb http://package.perforce.com/apt/ubuntu focal release
    \end{verbatim}
    \item Update machine and install Helix Core
    \begin{verbatim}
        $ sudo apt-get update
        $ sudo apt-get install helix-p4d
    \end{verbatim}
    \item Run configure file
    \begin{verbatim}
        $ sudo /opt/perforce/sbin/configure-helix-p4d.sh
    \end{verbatim}
    \item Download visual interface (p4v) from 
    \href{https://www.perforce.com/downloads/helix-visual-client-p4v}{\color{blue} Download Page}
    \item Additional config steps
    \begin{verbatim}
        $ export P4PORT=ssl:1666
        $ export P4USER=super
        # add the following line to ~/.bashrc
        export PATH=${PATH}:~/perforce/p4v-2023.3.2495381/bin
    \end{verbatim}
    \item Login and start p4v
    \begin{verbatim}
        $ p4 login
        $ p4v # launches GUI
    \end{verbatim}
\end{enumerate}
Having done the above steps, the GUI can be launched bringing up the following screen:
\begin{figure}[H]
    \centering
    \includegraphics[width=\textwidth]{p4vlogin20231029.png}
      \caption{p4v login}
      \label{fig:p4v login}
\end{figure}


\subsection{Development and version control}
...
