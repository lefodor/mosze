\subsection{Bring everything together}
\href{https://docs.godotengine.org/en/stable/getting_started/first_2d_game/05.the_main_game_scene.html}{\color{blue}The main game scene}.
Add main scene containing player and mob scenes. Additionally, this scene controlls the flow of the game by handling timers.
Create new scene by adding a \textit{Node} called \textbf{Main}. Then click on the \textbf{Instance} button (see cursor
on the image) and select \textit{player.tscn}.
\begin{figure}[H]
    \centering
    \includegraphics[scale=0.5]{main-scene.png}
    \caption{workspace after enemy creation}
    \label{fig:main-scene}
\end{figure}
Add three Timer objects to \textit{Main} and name them and set \textit{Wait Time} property as follows:
\begin{itemize}
    \item MobTimer - control mobs spawn: 0.5
    \item ScoreTimer - increment score by seconds: 1
    \item StartTimer - delay before starting: 2
\end{itemize}
Additionally, create a \textit{Marker2D} for the starting position of the player and set its \textit{Position} property
to 240, 450.
\subsubsection{Spawning mobs}
Enemies should be created randomly on the edge of the screen. To achieve this, add a \textit{Path2D} node and call it
\textbf{MobPath} as child of \textbf{Main}. Select the 4 corners of the screen to setup the path of the curve used
for spawning enemy objects. Finally, a \textit{PathFollow2D} node needs to be added as a child of \textbf{MobPath},
name it \textbf{MobSpawnLocation}. The scene is expected to look as below with the created curve:
\begin{figure}[H]
    \centering
    \includegraphics[width=\textwidth]{mobpath.png}
    \caption{mob path}
    \label{fig:mobpath}
\end{figure}
Add a script to the main scene with the following content:
\begin{verbatim}
    extends Node

    @export var mob_scene: PackedScene
    var score
    
    # Called when the node enters the scene tree for the first time.
    func _ready():
        pass #new_game()
    
    
    # Called every frame. 'delta' is the elapsed time since the previous frame.
    func _process(delta):
        pass
    
    
    func game_over():
        $ScoreTimer.stop()
        $MobTimer.stop()
        
    func new_game():
        score = 0
        $Player.start($StartPosition.position)
        $StartTimer.start()
    
    func _on_start_timer_timeout():
        $MobTimer.start()
        $ScoreTimer.start()
    
    func _on_score_timer_timeout():
        score += 1
    
    func _on_mob_timer_timeout():
        # Create a new instance of the Mob scene.
        var mob = mob_scene.instantiate()
    
        # Choose a random location on Path2D.
        var mob_spawn_location = get_node("MobPath/MobSpawnLocation")
        mob_spawn_location.progress_ratio = randf()
    
        # Set the mob's direction perpendicular to the path direction.
        var direction = mob_spawn_location.rotation + PI / 2
    
        # Set the mob's position to a random location.
        mob.position = mob_spawn_location.position
    
        # Add some randomness to the direction.
        direction += randf_range(-PI / 4, PI / 4)
        mob.rotation = direction
    
        # Choose the velocity for the mob.
        var velocity = Vector2(randf_range(150.0, 250.0), 0.0)
        mob.linear_velocity = velocity.rotated(direction)
    
        # Spawn the mob by adding it to the Main scene.
        add_child(mob)
\end{verbatim}
The \textit{Main} node needs to be able to access variables from \textit{Mob} scene. For this, the 
\textit{Mob Scene} {$<$}empty{$>$} button in the \textit{Inspector} tab and load \textit{mob.tscn} in the pop-up window.
\begin{figure}[H]
    \centering
    \includegraphics[scale=0.5]{mob-scene-variables.png}
    \caption{mob scene variables}
    \label{fig:mob-scene-variables}
\end{figure}
Sort out signals:
\begin{itemize}
    \item Connect the \textit{hit} signal of the player scene with a receiver method called \textbf{game\_over}.
    \item Connect \textit{timeout()} signal of each Timer nodes to the main script.
\end{itemize}
Now, pressing F5 should run the game with a player character disappearing when being hit by an enemy.
\subsection{Version control of final steps}
\begin{figure}[H]
    \centering
    \includegraphics[width=\textwidth]{copy-merge-main.png}
    \caption{copy merge main}
    \label{fig:copy-merge-main}
\end{figure}

\begin{figure}[H]
    \centering
    \includegraphics[width=\textwidth]{show-main-dev-conflict.png}
    \caption{show main dev conflict}
    \label{fig:show-main-dev-conflict}
\end{figure}

\begin{figure}[H]
    \centering
    \includegraphics[width=\textwidth]{resolve-main-dev-conflict.png}
    \caption{resolve main dev conflict}
    \label{fig:resolve-main-dev-conflict}
\end{figure}

\begin{figure}[H]
    \centering
    \includegraphics[width=\textwidth]{after-merge-conflicts-resolved.png}
    \caption{after merge conflicts resolved}
    \label{fig:after-merge-conflicts-resolved}
\end{figure}

\begin{figure}[H]
    \centering
    \includegraphics[width=\textwidth]{final-dev1-history.png}
    \caption{final dev1 history}
    \label{fig:final-dev1-history}
\end{figure}

\begin{figure}[H]
    \centering
    \includegraphics[width=\textwidth]{final-main-history.png}
    \caption{final main history}
    \label{fig:final-main-history}
\end{figure}